\documentclass[12pt]{article}

\usepackage{amssymb, amsmath, lineno}
\usepackage[margin=1in]{geometry}
\title{Pokemon Go Efficiency}
\author{Arjun Biddanda}
\date{\today}
% Removes the page numbering
\pagenumbering{gobble}

\begin{document}
\maketitle
\linenumbers

\subsection*{Problem}

Count Von Count, the counting count on “Sesame Street,” counts aloud on Twitter. If he counts up by one with each tweet — “One!” “Two!” “Three!” … “Five hundred thirty eight!” etc. — how high can he go before hitting the 140-character limit? Note: The count is enthusiastic and must end all of his tweets with an exclamation point

\subsection*{Solution + Computation}

The brute force method here takes a very, very long time (by design) so it takes forever to compute out a number high enough to satisfy this problems answer so we need to take a different approach.

A valid approach to this would be to approach this similar to a dynamic programming problem, where we keep track of the maximum number of characters at each potential place (e.g. ones, tens, hundreds, etc...) and work from that particular maximum.

From the computational angle, the dynamic program to solve this problem (very generally is in the \texttt{src} directory) and the solution when run with a character limit of $140$ is $373373373373$ which is the first number to hit the limit. 


\end{document}
