\documentclass[12pt]{article}

\usepackage{amssymb, amsmath, lineno, graphicx, float}
\usepackage[margin=1in]{geometry}
\title{The 100-sided Dice Problem}
\author{Arjun Biddanda}
\date{\today}
\pagenumbering{gobble}

\begin{document}
\maketitle
\linenumbers

\subsection*{Problem}

At the beginning of time, there is a single microorganism. Each day, every member of this species either splits into two copies of itself or dies. If the probability of multiplication is $p$, what are the chances that this species goes extinct?


\subsection*{Solution}

We first need to create an equation for the time-dependent population size of the microorganism, which ends up being recursive in nature.

We can try to first look at a couple of cases:

$$
N(1) =
  \begin{cases}
      2 &, p \\
      0 & (1-p)\\
   \end{cases}
$$

Then for the second generation:

$$
(N(2) | N(1) = 2) =
  \begin{cases}
      4 &, p^2 \\
      2 &, p(1-p)\\
      0 & (1-p)^2\\
   \end{cases}
$$

% In many ways, this is similar to the wright-fisher model in population genetics


\end{document}
